\section{Grundlagen} %%Hier müssten wir eigentlich eine Unterkategorie mit MOOC und ISSuccess einführen...
\label{sec:grundlagen}
<<<<<<< HEAD
<<<<<<< HEAD
Es war das Jahr 2008, als George Siemens und Stephen Downes an der Universität Manitoba in Kanada eine Vorlesung über das Internet verbreiteten und dabei über 2200 Teilnehmer von der ganzen Welt erreichen konnten. Dieses Ereignis ging als der erste Massive Open Online Course (im folgenden MOOC genannt) in die Geschichte ein. Jeder Interessierte konnte damals ohne jegliche Kosten an dem Kurs teilnehmen. Diese Idee entwickelte sich in den folgenden Jahren weiter und ab 2011 begannen die ersten US-amerikanischen Universitäten den eigenen Studierenden MOOCs als Erweiterungsangebot anzubieten. Seit dem bieten die meisten Universitäten eigene MOOCs an, welche sich dabei meist an externe Personen richten. Darüber hinaus gibt es unterschiedliche Unternehmen, die eigene MOOCs anbieten. In der Regel ist die Teilnahme an einem MOOC kostenlos, die Ausstellung eines Zertifikats inklusive Credit Points nach erfolgreicher Teilnahme jedoch nur gegen Gebühr möglich.
\newline 
Auch die Leuphana Universität Lüneburg bietet mit der Digital School Massive Open Online Courses an. Dabei werden unterschiedliche Kooperationspartner wie unter anderem das Goethe-Institut mit einbezogen.
=======
Es war das Jahr 2008, als George Siemens und Stephen Downes an der Universität Manitoba in Kanada eine Vorlesung über das Internet verbreiteten und dabei über 2200 Teilnehmer von der ganzen Welt erreichen konnten. Dieses Ereignis ging als der erste Massve Open Online Course (im folgenden MOOC genannt) in die Geschichte ein. Jeder Interessierte konnte damals ohne jegliche Kosten an dem Kurs teilnehmen. Diese Idee entwickelte sich in den folgenden Jahren weiter und ab 2011 begannen die ersten US-amerikanischen Universitäten den eigenen Studierenden MOOCs als Erweiterungsangebot anzubieten. Seit dem bieten die meisten Universitäten eigene MOOCs an, welche sich dabei meist an externe Personen richten. Darüber hinaus gibt es unterschiedliche Unternehmen, die eigene MOOCs anbieten. In der Regel ist die Teilnahme an einem MOOC kostenlos, die Ausstellung eines Zertifikats inklusive Credit Points nach erfolgreicher Teilnahme jedoch nur gegen Gebühr möglich.\newline Auch die Leuphana Universität Lüneburg bietet mit der Digital School Massive Open Online Courses an. Dabei werden unterschiedliche Kooperationspartner wie unter Anderem das Goethe-Institut mit einbezogen.
>>>>>>> parent of 9224982... hotfix MOOC
=======
Es war das Jahr 2008, als George Siemens und Stephen Downes an der Universität Manitoba in Kanada eine Vorlesung über das Internet verbreiteten und dabei über 2200 Teilnehmer von der ganzen Welt erreichen konnten. Dieses Ereignis ging als der erste Massve Open Online Course (im folgenden MOOC genannt) in die Geschichte ein. Jeder Interessierte konnte damals ohne jegliche Kosten an dem Kurs teilnehmen. Diese Idee entwickelte sich in den folgenden Jahren weiter und ab 2011 begannen die ersten US-amerikanischen Universitäten den eigenen Studierenden MOOCs als Erweiterungsangebot anzubieten. Seit dem bieten die meisten Universitäten eigene MOOCs an, welche sich dabei meist an externe Personen richten. Darüber hinaus gibt es unterschiedliche Unternehmen, die eigene MOOCs anbieten. In der Regel ist die Teilnahme an einem MOOC kostenlos, die Ausstellung eines Zertifikats inklusive Credit Points nach erfolgreicher Teilnahme jedoch nur gegen Gebühr möglich.\newline Auch die Leuphana Universität Lüneburg bietet mit der Digital School Massive Open Online Courses an. Dabei werden unterschiedliche Kooperationspartner wie unter Anderem das Goethe-Institut mit einbezogen.
>>>>>>> parent of 9224982... hotfix MOOC



%\cite{king2006meta} sagten:" \blindtext" \parencite{behrenbruch2013understanding}