\section{Einleitung}
\label{sec:einleitung}

Die Vermittlung von Wissen in einem universitären Kontext beschränkte sich bis zu der flächendeckenden Verbreitung des Internets in Privathaushalten auf die lokale Anwesenheit in Vorlesungen, Seminaren und Kursen. Die ersten e-Learning Angebote konnten jedoch erstmals auch unabhängig von örtlicher Gebundenheit genutzt werden. Massive Open Online Courses (MOOC) hingegen ermöglichen die Verlagerung ganzer Veranstaltungen ins Internet und unterscheiden sich somit zu klassischen e-Learning Angeboten, die meist nur zu einem bestimmten Themengebiet medial aufbereitete Erklärungen bieten.
\newline
In der wissenschaftlichen Fachliteratur sind bereits einige Untersuchungen über den Erfolg von e-Learnings durchgeführt worden. Im Bereich von MOOCs hingegen sind eher wenige Studien vorhanden. Gerade die Frage, wodurch der Erfolg der einzelnen Teilnehmer eines MOOCs entscheidend beeinflusst wird, wurde bisher unzureichend untersucht. Um dieser Fragestellung nachzugehen, sind deutliche Parallelen zu e-Learnings festzustellen. Es bietet sich daher an, Teile von existierenden Forschungsmodellen auf diesen Bereich anzuwenden. Die Datenbasis ergibt sich aus beantworteten Fragebögen, die zu unterschiedlichen Zeiten an Teilnehmer eines MOOCs der Leuphana Digital School versendet wurden. 