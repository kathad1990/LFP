\section{Ergebnisse}
\label{sec:ergebnisse}
\nocite{lohmoller2013latent}
Zur Evaluierung des Forschungsmodells und der Hypothesen wurde die Partial-Least-Square(PLS)-Methode verwendet. Dabei werden  "`die Modellparameter so geschätzt, dass der Anteil der erklärten Varianz der abhängigen Variable und der Indikatoren eines reflektiv gemessenen Konstrukts maximiert wird."'\parencite[S.16]{nitzl2010anwenderorientierte} Ein besonderer Vorteil der PLS-Methode ist ihre Anwendungsmöglichkeit auch bei verhältnismäßig kleiner Stichprobengröße. Zur Kalkulation einer minimalen Stichprobengröße kommt häufig eine Faustregel zur Anwendung, nach der die Stichprobengröße mindestens das zehnfache des Konstruktes mit der größten Anzahl zu schätzender Parameter sein sollte.\parencite[vgl.][S.394]{islam2013investigating} Dieses Kriterium wird in dieser Studie erfüllt. Zur Analyse wurde die Softwareapplikation SmartPLS verwendet. 

Für die Modellbeurteilung wird zunächst das reflektive Messmodell einer Güteprüfung unterzogen. Die Konvergenzvalidität wird anhand der Kriterien Indikatorreliabilität, Konstruktreliabilität und duchschnittlich erfassten Varianz (DEV) kritisch betrachtet, während die Validität mithilfe der Diskriminanzvalidität überprüft wird.

Die Indikatorreliabiltät 

Die Konstruktreliabilität $\rho$ untersucht unter Einsatz der internen Konsistenz (auch composite reliability genannt) "`wie gut die Indikatoren eine latente Variable wiedergeben."' Ein Wert von $\rho$ $\geq$ 0,6 gilt als aktzeptabel.\parencite[vgl.][S.212]{ringle2007beurteilung} Das Messmodell weist $\rho$ > ... und liegen damit über den Schwellenwert. 

Die durchschnittliche erfasste Varianz (DEV) "`setzt den Anteil der erklärten Varianz in Relation zum Messfehler einer latenten Variable."' \parencite[S.25]{nitzl2010anwenderorientierte} Ein Wert von DEV $\geq$ 0,5 stellt einen ausreichend hohen Wert dar. Die DEV liegt in dieser Studie mit DEV > ... ebenfalls über dem genannten Schwellenwert und ist somit akzeptabel. 

Die Diskriminanzvalidilität hingegen "`gibt an, in welchem Ausmaß sich die Indikatoren eines Konstrukts von denen eines anderen Konstrukts unterscheiden."'\parencite[S.26]{nitzl2010anwenderorientierte}. Zur Überprüfung der Diskriminanzvalididtät kann das Fornell-Lacker-Kriterium und die Cross Loadings herangezogen werden. 
Bei ersterem wird die Wurzel der DEV einer latenten Variable verglichen mit jeder Korrelation dieser latenten Variable mit einer anderen latenten Variablen und sollte stets größer sein.\parencite[vgl.][S.26]{nitzl2010anwenderorientierte} Das Fornell-Lacker-Kriterium wird in dieser Studie erfüllt. (Siehe Tabelle ...) Die Cross Loadings können Tabelle ... entnommen werden. Ein Indikator sollte dabei die stärkste Beziehung mit dem ihm zugeordneten Konstrukt aufweisen \parencite[vgl.][S.26]{nitzl2010anwenderorientierte}, was ebenfalls erfüllt ist.   \nocite{fornell1981evaluating}
Das Messmodell erfüllt damit alle Gütekriterien. Zur Beurteilung des Strukturmodells werden das Bestimmtheitsmaß R$^2$, die Pfadkoeeffizienten , die Effektstärke f$^2$ und die Prognoserelevanz Q$^2$ herangezogen.  

Die Werte für das Bestimmtheitsmaß R$^2$ sind in Tabelle ... enthalten. Das Bestimmtheitsmaß "`gibt den Anteil der erklärten Varianz im Verhältnis zur Gesamtvarianz an."'\parencite[S.32]{nitzl2010anwenderorientierte} Eine Einteilung relevanterer Schwellenwerte wurde von \cite[S.323]{chin1998partial} in einer Studie ermittelt. Die Werte für R$^2$ von 0,67, 0,33 und 0,19 wurden in "`substanziell"', "`mittelgut"' und "`schwach"' eingeteilt. In dieser Studie sind die R$^2$ dementsprechend als "`mittelgut"' einzustufen. (siehe Tabelle ...)

Die Pfadkoeeffizienten können Werte zwischen -1 und 1 annehmen. Ein Wert Nahe 0 gilt als schwach. Als akzeptabel wird ein Wert kleiner -0,2 oder größer 0,2 angesehen.\parencite[vgl.][S.11]{chin1998commentary} In der Studie können dann 


5.2. Exploratory and confirmatory analysis

Sample Selection
Demographic Statistics

Main statistics.
Fig. 2. Standard coefficients and significance values.

Table 7
The square root of AVE (italic at diagonal) and correlation coefficients.

Path coefficients and significances.












% Tabellenformat 1
\begin{table}[h] 
\caption{Zusammenfassung G\"utepr\"ufung}
\label{tab:Zusammenfassung} 
\begin{tabular}{|l|l|l|l|l|l|}
  \hline
\textbf{Factor} & \textbf{AVE} & \textbf{CR} & \textbf{R Square} & \textbf{Q Square} & \textbf{Alpha} \\
  \hline
 Net System Benefit & 1.000 & 1.000 & 0.503 & 0.476 & 1.000 \\
  \hline 
 Service Quality & 0.809 & 0.927 & & & 0.882 \\
  \hline
 System Quality & 0.730 & 0.890 & & & 0.817 \\
  \hline
 User Satisfication & 0.786 & 0.880 & 0.526 & 0.388 & 0.737 \\ 
 \hline
\end{tabular}	
\end{table}


% Tabellenformat 2
\begin{table}[h] 
\caption{Zusammenfassung G\"utepr\"ufung 2}
\label{tab:Zusammenfassung} 
\begin{tabular}{@{}llllll@{}} \toprule

\textbf{Factor} & \textbf{AVE} & \textbf{CR} & \textbf{R Square} & \textbf{Q Square} & \textbf{Alpha} \\ \midrule

 Net System Benefit & 1.000 & 1.000 & 0.503 & 0.476 & 1.000 \\
 
 Service Quality & 0.809 & 0.927 & & & 0.882 \\

 System Quality & 0.730 & 0.890 & & & 0.817 \\

 User Satisfication & 0.786 & 0.880 & 0.526 & 0.388 & 0.737 \\ \bottomrule
\end{tabular}	
\end{table}