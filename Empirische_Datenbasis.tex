\section{Empirische Datenbasis}
\label{sec:emp_daten}

Forschungsmethode

Erstellung des Fragebogens
Der Aufbau des Forschungsmodell lässt sich \ref{tab:Forschungsmodell} entnehmen. Aufgrund der internationalen Ausrichtung des Kurses wurden die Fragen in Englisch gestellt. 
Es wurde eine Seven-point Likert Skala verwendet. Die Antwortmöglichkeiten reichten von "Strongly disagree (1)” bis “Strongly Agree (7)"  
Die Datenerhebung fand zu drei unterschiedlichen Zeitpunkten statt: zu Beginn des Kurses (T1), während des Kurses (T2) und zum Ende des Kurses (T3). Die Beantwortung der Umfrage unterlag einer freiwilligen Basis. Die im Forschungsmodell beschriebenen Items sind in den Fragebögen T2 und T3 enthalten. \todo{ITEM Enriched knowledge genau benennen}.
Für die Erstellung des Modells wurden T2 und T3 verwendet, da die Teilnehmer zu Beginn des Kurses keine Angaben über ihren persönlichen Erfolg machen konnten (Frageitem: Enriched Knowledge fehlt). 
 
% Tabellenformat 2

\begin{table}[ht] 
\footnotesize
\caption{Forschungsmodell}
\label{tab:Forschungsmodell} 
\begin{tabular}{@{}lp{10cm}r@{}} \toprule

\textbf{Konstrukt} & \textbf{Item} & \textbf{Factorloadings} \\ \midrule

Servicequalität & The Leuphana Digital School provides a proper level of online assistance and explanation & 0.8\\ 
& The teaching staff is highly availability for consultation & 0.7 \\
& The teaching staff provides satisfactory support to users using Leuphana Digital School & 0.6 \\ 
Systemqualität & Leuphana Digital School’s technical system has attractive features to appeal to the users. & 0.8\\ 
& Leuphana Digital School’s technical system is easy to use. & 0.7 \\
& Leuphana Digital School’s technical system provides a personalized information presentation. & 0.6 \\ 
Nutzerzufriedenheit & Most of the users bring a positive attitude or evaluation towards Leuphana Digital School. & 0.8\\ 
& Leuphana Digital School’s technical system is easy to use. & 0.7 \\ 
Persönlicher Nutzen & Leuphana Digital School helps you think through problems.  & 1\\ 
& All in all, my knowledge has been enriched as a result of the course & 0.4 \\ \addlinespace 
  \bottomrule

\end{tabular}	
\end{table}


% Tabellenformat 2

\begin{table}[ht] 
\caption{Forschungsmodell2}
\label{tab:Forschungsmodell} 
\begin{tabular}{@{}lp{12cm}r@{}} \toprule

\textbf{Konstrukt} & \textbf{Item} \\ \midrule

Servicequalität & The Leuphana Digital School provides a proper level of online assistance and explanation \\ \addlinespace
& The teaching staff is highly availability for consultation\\\addlinespace
& \parbox[t]{12cm}{The teaching staff provides satisfactory support to users using Leuphana Digital School} \\ \addlinespace
Systemqualität & \parbox[t]{12cm}{Leuphana Digital School’s technical system has attractive features to appeal to the users.}\\ \addlinespace
& Leuphana Digital School’s technical system is easy to use. \\\addlinespace
& \parbox[t]{12cm}{Leuphana Digital School’s technical system provides a personalized information presentation.} \\ \addlinespace
Nutzerzufriedenheit & \parbox[t]{12cm}{Most of the users bring a positive attitude or evaluation towards Leuphana Digital School.} \\ \addlinespace
& Leuphana Digital School’s technical system is easy to use.  \\ \addlinespace 
Persönlicher Nutzen & \parbox[t]{12cm}{Leuphana Digital School helps you think through problems.} \\ \addlinespace 
& \parbox[t]{12cm}{All in all, my knowledge has been enriched as a result of the course (nur in Fragebogen 2 und 3)} \\ \addlinespace 
  \bottomrule

\end{tabular}	
\end{table}

Teilnehmer


Die vorliegenden Daten entstammen aus Befragungen der Teilnehmer des MOOCs (Mentored Open Online Course) Psychology of Negotiations - Reaching Sustainable Agreements in Negotiations on "Commons" der Leuphana Digital School. Der Kurs fand von Mai bis August 2014 statt und war offen für Teilnehmer aus der ganzen Welt. Die Teilnehmerzahl wurde auf 1000 begrenzt um die Qualität der Beratung und Führung durch Mentoren und Lehrende zu gewährleisten. Der Kurs war gebührenfrei. Bei erfolgreichem Abschluss konnten   Teilnehmer - gegen eine Gebühr von 20 	\texteuro - ein Zertifikat der Universität erhalten und 5 Credit Points (ECTS) zugeschrieben bekommen, welche dem eigenen Studium angerechnet werden konnten.   
Die Fragebögen wurden an alle Teilnehmer verschickt. Da die Anzahl der Kursteilnehmer, die den Fragebogen erhalten haben unbekannt ist, kann die Returnquote nicht angegeben werden. Gemessen an den Antworten, ist diese jedoch eher gering - vor allem beim zweiten und dritten Fragebogen. Bei Ersterem  lagen 32 Antworten vor, nach der Bereinigung von ungültigen oder unvollständigen Antworten lagen 29 verwertbare Antworten vor. Die Bereinigung ungültiger und unvollständiger Antworten reduzierte die nutzbaren Antworten im dritten Fragebogen von 48 auf 36.
Die Demographischen Daten der Fragebögen T2 und T3 sind in \ref{tab:Demographische Daten} gegeben. 


\begin{table}[ht] 
\caption{Demographische Daten}
\label{tab:Demographische Daten} 
\begin{tabular}{@{}lp{5cm}r@{}} \toprule

 & \textbf{Anzahl}&\textbf{in Prozent} \\ \midrule

\textit{Geschlecht}		& 				& \\ 
weiblich 				&  40 			& 62 \\
männlich				&  25			& 38 \\ 
Total					&  65			& 100 \\
\textit{Alter}			& 				&   \\
21-30 Jahre				&  35			& 54 \\
31-40 Jahre				&  10			& 15	  \\
> 41 Jahre				&  20 			& 31 \\
Total 					&  65			& 100 \\
\textit{Herkunft}		&				&   \\
Deutschland				& 27 			& 42  \\
Europa (excl. Deutschland) &18			& 28  \\
Afrika 					& 6				& 9   \\
Asien 					& 6				& 9   \\
Nordamerika				& 4				& 6   \\
Südamerika				& 4				& 6	 \\
Total					& 65				& 100  \\ 
  \bottomrule

\end{tabular}	
\end{table}


Data Analysis Technique
Instrument validation
We followed the procedures outlined by Gefen and Straub (2005) to test discriminant and convergent validity. Discriminant validity refers to whether the items measure the construct in question or other (related) constructs (Gefen and Straub, 2005). We verified discriminant validity using correlation matrix and factor analysis. Table 4 shows the correlation matrix with the square root of average variance extracted (AVE) values presented diagonally. The square root of the AVE value for the variables is consistently greater than the off-diagonal correlation values, suggesting satisfactory discriminant validity between the variables (Fornell and Larcker, 1981)

Zur Analyse im Rahmen dieser Arbeit wurde auf Strukturgleichungsmodellierung zurückgegriffen, 
For this paper, structural equation modeling (SEM) using partial least squares (PLS) was used to evaluate the research model and hypotheses.



