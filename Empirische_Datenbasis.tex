\section{Empirische Datenbasis}
\label{sec:emp_daten}

Forschungsmethode

Erstellung des Fragebogens
Der Aufbau des Forschungsmodell lässt sich \ref{tab:Forschungsmodell} entnehmen. Aufgrund der internationalen Ausrichtung des Kurses wurden die Fragen in Englisch gestellt. 
Es wurde eine seven-point-Likert Skala verwendet. Die Antwortmöglichkeiten reichten von "Strongly disagree (1)” bis “Strongly Agree (7)" 
 
Die Datenerhebung fand zu drei unterschiedlichen Zeitpunkten statt: zu Beginn des Kurses, während des Kurses und zum Ende des Kurses. Die Beantwortung der Umfrage fand auf freiwilliger Basis statt. Die im Forschungsmodell beschriebenen Items sind in allen Fragebögen enthalten. \todo{ITEM Enriched knowledge genau benennen}.

Für die Erstellung des Modells wurden zunächst nur Daten des 2. und 3. Fragebogens verwendet, da die Teilnehmer zu Beginn keine Angaben über die Verbesserung ihres Wissensstandes machen konnten und der Persönliche Nutzen beim ersten Fragebogen lediglich aus einem Item bestehen würde. 
 
% Tabellenformat 2

\begin{table}[ht] 
\footnotesize
\caption{Forschungsmodell}
\label{tab:Forschungsmodell} 
\begin{tabular}{@{}llr@{}} \toprule

\textbf{Konstrukt} & \textbf{Item} & \textbf{Factorloadings} \\ \midrule

Servicequalität & \parbox[t]{10cm}{The Leuphana Digital School provides a proper level of online assistance and explanation} & 0.8\\ \addlinespace
& The teaching staff is highly availability for consultation & 0.7 \\\addlinespace
& \parbox[t]{10cm}{The teaching staff provides satisfactory support to users using Leuphana Digital School} & 0.6 \\ \addlinespace
Systemqualität & \parbox[t]{10cm}{Leuphana Digital School’s technical system has attractive features to appeal to the users.} & 0.8\\ \addlinespace
& Leuphana Digital School’s technical system is easy to use. & 0.7 \\\addlinespace
& \parbox[t]{10cm}{Leuphana Digital School’s technical system provides a personalized information presentation.} & 0.6 \\ \addlinespace
Nutzerzufriedenheit & \parbox[t]{10cm}{Most of the users bring a positive attitude or evaluation towards Leuphana Digital School.} & 0.8\\ \addlinespace
& Leuphana Digital School’s technical system is easy to use. & 0.7 \\ \addlinespace 
Persönlicher Nutzen & \parbox[t]{10cm}{Leuphana Digital School helps you think through problems.}  & 1\\ \addlinespace 
& \parbox[t]{10cm}{All in all, my knowledge has been enriched as a result of the course} & 0.4 \\ \addlinespace 
  \bottomrule

\end{tabular}	
\end{table}


% Tabellenformat 2

\begin{table}[ht] 
\caption{Forschungsmodell2}
\label{tab:Forschungsmodell} 
\begin{tabular}{@{}llr@{}} \toprule

\textbf{Konstrukt} & \textbf{Item} \\ \midrule

Servicequalität & \parbox[t]{12cm}{The Leuphana Digital School provides a proper level of online assistance and explanation} \\ \addlinespace
& The teaching staff is highly availability for consultation\\\addlinespace
& \parbox[t]{12cm}{The teaching staff provides satisfactory support to users using Leuphana Digital School} \\ \addlinespace
Systemqualität & \parbox[t]{12cm}{Leuphana Digital School’s technical system has attractive features to appeal to the users.}\\ \addlinespace
& Leuphana Digital School’s technical system is easy to use. \\\addlinespace
& \parbox[t]{12cm}{Leuphana Digital School’s technical system provides a personalized information presentation.} \\ \addlinespace
Nutzerzufriedenheit & \parbox[t]{12cm}{Most of the users bring a positive attitude or evaluation towards Leuphana Digital School.} \\ \addlinespace
& Leuphana Digital School’s technical system is easy to use.  \\ \addlinespace 
Persönlicher Nutzen & \parbox[t]{12cm}{Leuphana Digital School helps you think through problems.} \\ \addlinespace 
& \parbox[t]{12cm}{All in all, my knowledge has been enriched as a result of the course (nur in Fragebogen 2 und 3)} \\ \addlinespace 
  \bottomrule

\end{tabular}	
\end{table}

Teilnehmer


Die vorliegenden Daten entstammen aus Befragungen der Teilnehmer des MOOCs (Mentored Open Online Course) Psychology of Negotiations - Reaching Sustainable Agreements in Negotiations on "Commons" der Leuphana Digital School. Der Kurs fand von Mai bis August 2014 statt und war offen für Teilnehmer aus der ganzen Welt. Die Teilnehmerzahl wurde auf 1000 begrenzt um die Qualität der Beratung und Führung durch Mentoren und Lehrende zu gewährleisten. Der Kurs war gebührenfrei. Bei erfolgreichem Abschluss konnten   Teilnehmer - gegen eine Gebühr von 20 	\texteuro - ein Zertifikat der Universität erhalten und 5 Credit Points (ECTS) zugeschrieben bekommen, welche dem eigenen Studium angerechnet werden konnten.   


Table 3
Demographic information.
Gender
Age
Country


