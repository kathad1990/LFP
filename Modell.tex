\section{Modell}
\label{sec:modell}
Das IS Success Modell wurde als Basis genommen, um die die Einflussfaktoren eines MOOCs zu analysieren. Als relevante Konstrukte wurde dabei Systemqualität, Servicequalität, Nutzerzufriedenheit und Net Benefits identifiziert.  

In der Vergangenheit konnte bereits in diversen Studien zu IS Success-Modellen - unter anderen im e-Learning-Bereich - die Bedeutung der Systemqualität auf die Nutzerzufriedenheit bestätigt werden (Alsabawy et al., 2013; Motaghian et al., 2013; Saba, 2013; Tajuddin et al., 2013; Hassanzadeh et al., 2012; Kim et al., 2012; Islam, 2012; Wang and Chiu, 2011; Rai et al., 2009; Wu et al., 2008), daher gilt auch hier die Hypothese: \medskip

\textbf{H1:} \textit{Die Systemqualität hat einen positiven Einfluss auf die Nutzerzufriedenheit.} \medskip

Die Servicequalität im IS Success-Modell wurde von manchen Autoren in der Literatur kritisch betrachtet \todo{Quellen, kritisch gegenüber Servicequalität)}. Dabei wurde unter Servicequalität vorwiegend der Support verstanden, den Nutzer vom IS erhalten konnten, wie z.B. Training (\parencite{petter2009meta}) oder Helpdesk. Im Rahmen eines MOOCs wird unter Servicequalität allerdings vorwiegend die (Online-)Betreuung durch Mentoren, Lehrende und Mitarbeiter des MOOC-Anbieters verstanden. Aus diesem Grund wird der Servicequalität ein maßgeblicher Einfluss auf die Nutzerzufriedenheit eingeräumt: \medskip

\textbf{H2:} \textit{Die Servicequalität hat einen positiven Einfluss auf die Nutzerzufriedenheit.}\medskip

Der Term "`Net Benefit"' wird von dem neuformulierten IS Success Modell übernommen. Er besagt, dass es sich um ein positives Ergebnis (Benefit) handelt, was allerdings auch negative Einflüsse enthalten kann (Net) \parencite[vgl.][S.2974]{delone2002information}. Der Erfolg eines MOOCs kann aus verschiedenen Perspektiven gemessen werden, z.B. aus Sicht des Teilnehmers, des MOOC Anbieters, der Lehrenden, des Systemanbieters etc. Jede Perspektive hätte eine andere Definition von Net Benefit zur Folge. In erster Linie sollte der Erfolg eines MOOCs an den Teilnehmern sichtbar sein, daher wird in dieser Studie der Erfolg des MOOCs aus der individuellen Sicht des Teilnehmers gemessen. Eine hohe Nutzerzufriedenheit wurde in vergangenen Studien bereits als Einflussgröße für den Net Benefit ermittelt (\todo{Quellen Studien} und stimmt mit dem IS Success Modell überein. Die empirische Datenerhebung fand auf Basis eines MOOCs der Leuphana Digital School (LDS) statt (siehe \nameref{sec:emp_daten}). Die LDS definiert MOOC als "`Mentored Open Online Course"' und weist der Servicequalität damit eine besondere Bedeutung zu. Daher wird in diesem Rahmen - im Gegensatz zum herkömmlichen IS Success Modell - auch von einem positiven Einfluss der Servicequalität auf den Net Benefit ausgegangen. Die weiteren Hypothesen lauten entsprechend: \medskip


\textbf{H3:} \textit{Die Servicequalität hat einen positiven Einfluss auf den Net Benefit.} 


\textbf{H4:} \textit{Die Nutzerzufriedenheit hat einen positiven Einfluss auf den Net Benefit.} \medskip


Die Konstrukte Informationsqualität und Nutzung, bzw. beabsichtigte Nutzung werden - unter anderem aus Vereinfachungsgründen - in dem hier behandelten Modell nicht berücksichtigt. Das Wegfallen des Konstruktes Nutzung folgt dabei der Argumentation von \cite{seddon1997respecification}, der angibt, dass Nutzung keine Erfolgsvariable in einem Kausalmodell darstellt. 

Daher 
Das im Rahmen dieses Papers verwendete IS Success Modell beinhaltet allerdings nur 4 Konstrukte, da erstens die Anzahl der Fragen streng reglementiert war und darüber hinaus auch nur eine geringe Response Rate erwartet wurde. Somit ist es mit dem kleinen sample besser möglich, qualitativ gute Aussagen zu treffen. Die Konstrukte sind in Tabelle \ref{tab:Forschungsmodell} genau definiert.

\todo{Warum andere Konstrukte ausgeschlossen?? Inhaltliche Erklärung}

Table 2
Definitions of dimensions.

% Tabellenformat 2
\begin{table}[ht] 
\footnotesize
\caption{Definition der Dimensionen}
\label{tab:Dimensionen} 
\begin{tabular}{@{}lp{9cm}l@{}} \toprule

\textbf{Konstrukt} & \textbf{Definiton} & \textbf{Quelle} \\ \midrule


Servicequalität 	& Qualitätsfaktor für die erwarteten Support,den die Nutzer in Anspruch nehmen können & \parbox[t]{4cm}{\cite{petter2008measuring}}\\ 

Systemqualität 		& Die erwarteten Eigenschaften und Funktionen von dem System & \parbox[t]{4cm}{\cite{petter2008measuring}}\\ 

Nutzerzufriedenheit & Das Ausmaß darüber, in wie weit die Bedürfnisse, Ziele und Wünsche während des MOOC erfüllt werden & \parbox[t]{4cm}{\cite{sanchez2009moderating}}\\ 

Net Benefit 		& Drückt aus, inwieweit Informationssysteme zum Erfolg einzelner Personen, Gruppen und Organisation beitragen können  & \parbox[t]{4cm}{\cite{conf/gi/GemlikNSB10} \cite{Petter:0aa} }\\ \addlinespace 
  \bottomrule

\end{tabular}	
\end{table}


