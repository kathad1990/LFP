\section{Modell}
\label{sec:modell}
In diversen Studien wurde das IS Success Modell bereits diskutiert und analysiert. 

Technical system quality has been found to have a significant positive effect on satisfaction in e-learning context (Alsabawy et al., 2013; Motaghian et al., 2013; Saba, 2013; Tajuddin et al., 2013; Hassanzadeh et al., 2012; Kim et al., 2012; Islam, 2012; Wang and Chiu, 2011; Rai et al., 2009; Wu et al., 2008)
In this section, the research variables and hypotheses are presented.

Hypothesen 
In diesem Modell werden drei 

In der Vergangenheit konnte bereits in diversen Studien zu IS Success-Modellen die Bedeutung der Systemqualität auf die Nutzerzufriedenheit bestätigt werden (\todo{diverse Quelle nennen}, daher gilt auch hier die Hypothese: 
  
\textbf{H1:} \textit{Die Systemqualität hat einen positiven Einfluss auf die Nutzerzufriedenheit.}


\textbf{H2:} \textit{Die Servicequalität hat einen positiven Einfluss auf die Nutzerzufriedenheit.}

\textbf{H3:} \textit{Die Nutzerzufriedenheit hat einen positiven Einfluss auf den Net Benefit.} 



Table 2
Definitions of dimensions.

% Tabellenformat 2
\begin{table}[ht] 
\footnotesize
\caption{Definition der Dimensionen}
\label{tab:Dimensionen} 
\begin{tabular}{@{}lp{9cm}l@{}} \toprule

\textbf{Konstrukt} & \textbf{Definiton} & \textbf{Quelle} \\ \midrule



Servicequalität & Qualitätsfaktor für die erwarteten Support,den die Nutzer in Anspruch nehmen können & \parbox[t]{4cm}{\cite{petter2008measuring}}\\ 
Systemqualität & Die erwarteten Eigenschaften und Funktionen von dem System & \parbox[t]{4cm}{\cite{petter2008measuring}}\\ Nutzerzufriedenheit & Das Ausmaß darüber, in wie weit die Bedürfnisse, Ziele und Wünsche während des MOOC erfüllt werden & \parbox[t]{4cm}{\cite{sanchez2009moderating}}\\ 
Net Benefit & \parbox[t]{9cm}{Drück aus, in wie weit Informationssysteme zum Erfolg einzelner Personen, Gruppen und Organisation beitragen können}  & \parbox[t]{4cm}{\cite{conf/gi/GemlikNSB10} \cite{Petter:0aa} }\\ \addlinespace 
  \bottomrule

\end{tabular}	
\end{table}


